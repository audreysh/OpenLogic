% Part: lambda-calculus
% Chapter: introduction
% Section: fixed-point-combinator

\documentclass[../../../include/open-logic-section]{subfiles}

\begin{document}

\olfileid{lam}{int}{fix}
\olsection{Fixed-Point Combinators}

Suppose you have a lambda term $g$, and you want another term $k$ with
the property that $k$ is $\beta$-equivalent to $gk$. Define terms
\[
\fn{diag}(x) = xx
\]
and
\[
l(x) = g(\fn{diag}(x))
\]
using our notational conventions; in other words, $l$ is the term
$\lambd[x][g(xx)]$. Let $k$ be the term $ll$. Then we have
\begin{align*}
k & = (\lambd[x][g(xx)])(\lambd[x][g(xx)]) \\
& \red  g((\lambd[x][g(xx)])(\lambd[x][g(xx)])) \\
& = gk.
\end{align*}
If one takes
\[
Y = \lambd[g][((\lambd[x][g(xx)])(\lambd[x][g(xx)]))]
\]
then $Yg$ and $g(Yg)$ reduce to a common term; so $Yg \equiv_\beta
g(Yg)$. This is known as ``Curry's combinator.'' If instead one takes
\[
Y = (\lambd[xg][g(xxg)])(\lambd[xg][g(xxg)])
\]
then in fact $Yg$ reduces to $g(Yg)$, which is a stronger statement.
This latter version of $Y$ is known as ``Turing's combinator.''

\end{document}
