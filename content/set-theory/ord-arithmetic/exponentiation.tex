\documentclass[../../../include/open-logic-section]{subfiles}

\begin{document}

\olfileid{sth}{ord-arithmetic}{expo}
\olsection{Ordinal Exponentiation}

We now move to ordinal exponentiation. Sadly, there is no \emph{nice}
synthetic definition for ordinal exponentiation.

Sure, there \emph{are} explicit synthetic definitions. Here is one.
Let $\text{finfun}(\alpha,\beta)$ be the set of all functions $f
\colon \alpha \to \beta$ such that $\Setabs{\gamma \in
\alpha}{f(\gamma) \neq 0}$ is equinumerous with some natural number.
Define a well-ordering on $\text{finfun}(\alpha,\beta)$ by $f
\sqsubset g$ iff $f \neq g$ and $f(\gamma_0) < g(\gamma_0)$, where
$\gamma_0 = \text{max}\Setabs{\gamma \in \alpha}{f(\gamma) \neq
g(\gamma)}$. Then we can define $\ordexpo{\alpha}{\beta}$ as
$\ordtype{\text{finfun}(\alpha, \beta), \sqsubset}$. Potter employs
this explicit definition, and then immediately explains:
\begin{quote}
	The choice of this ordering is determined purely by our desire to
	obtain a definition of ordinal exponentiation which obeys the
	appropriate recursive condition\ldots, and it is much harder to
	picture than either the ordered sum or the ordered product.
	\citep[p.~199]{Potter2004}
\end{quote}
Quite. We explained addition as ``a copy of $\alpha$ followed by a
copy of $\beta$'', and multiplication as ``a $\beta$-sequence of
copies of $\alpha$''. But we have nothing pithy to say about
$\text{finfun}(\alpha, \gamma)$. So instead, we'll offer the
definition of ordinal exponentiation just \emph{by} transfinite
recursion, i.e.:

\begin{defn}\ollabel{ordexporecursion}
\begin{align*}
	\ordexpo{\alpha}{0} &= 1\\
	\ordexpo{\alpha}{\beta\ordplus 1} &=\ordexpo{\alpha}{\beta} \ordtimes \alpha\\
	\ordexpo{\alpha}{\beta} &= \bigcup_{\delta < \beta}\ordexpo{\alpha}{\delta}& & \text{when $\beta$ is a limit ordinal}
\end{align*}
\end{defn}

If we were working \emph{as} set theorists, we might want to explore
some of the properties of ordinal exponentiation. But we have nothing
much more to add, except to note the unsurprising fact that ordinal
exponentiation does not commute. Thus $\ordexpo{2}{\omega} =
\bigcup_{\delta < \omega}\ordexpo{2}{\delta} = \omega$, whereas
$\ordexpo{\omega}{2} = \omega \ordtimes \omega$. But then, we should
not \emph{expect} exponentiation to commute, since it does not commute
with natural numbers: $\ordexpo{2}{3} = 8 < 9 = \ordexpo{3}{2}$. 

\begin{prob}
Using Transfinite Induction, prove that, if we define
$\ordexpo{\alpha}{\beta} = \ordtype{\text{finfun}(\alpha, \beta),
\sqsubset}$, we obtain the recursion equations of
\olref[sth][ord-arithmetic][expo]{ordexporecursion}.
\end{prob}

\end{document}