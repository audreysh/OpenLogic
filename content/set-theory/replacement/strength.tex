\documentclass[../../../include/open-logic-section]{subfiles}

\begin{document}

\olfileid{sth}{replacement}{strength}
\olsection{The Strength of Replacement}

Replacement is the axiom which makes the difference between $\ZF$
and~$\Z$. We helped ourselves to it throughout
\crefrange{sth:ordinals::chap}{sth:spine::chap}. In this chapter, we
will finally consider the question: is Replacement justified? To make
the question sharp, it is worth observing that Replacement is really
rather \emph{strong}.

Unless we go beyond $\Z$, we cannot prove the existence of any von Neumann
ordinal greater than or equal to $\omega + \omega$. Here is a sketch of
why. Working in~$\ZF$, consider the set $V_{\omega+\omega}$. This set acts
as the domain for a  \emph{model} for~$\Z$. Indeed, where $\phi$ is any
axiom of~$\Z$, let $\phi^{V_{\omega+\omega}}$ be the formula which results
by restricting all of $\phi$'s quantifiers to $V_{\omega+\omega}$ (that is,
replace ``$\lexists[x]$'' with ``$(\lexists[x \in V_{\omega+\omega}])$'', and
replace ``$\lforall[x]$'' with ``$(\lforall[x \in V_{\omega+\omega}])$''). It
can be shown that, for every axiom $\phi$ of~$\Z$, we have that $\ZF \vdash
\phi^{V_{\omega+\omega}}$. But $\omega+\omega$ is not \emph{in}
$V_{\omega+\omega}$, by  \olref[spine][rank]{ordsetrankalpha}. So $\Z$ is
consistent with the non-existence of $\omega+\omega$.

This is why we said, in \olref[ordinals][replacement]{sec}, that
\olref[ordinals][ordtype]{thmOrdinalRepresentation} cannot be proved
without Replacement. For it is easy, within~$\Z$, to define an
explicit well-ordering which intuitively \emph{should} have order-type
$\omega+\omega$. Indeed, we gave an informal example of this in
\olref[ordinals][idea]{sec}, when we presented the ordering on the
natural numbers given by:
\begin{align*}
	n \lessdot m \text{ iff }&\text{either $\left|n - m\right|$ is even and $n < m$},\\
	& \text{or $n$ is even and $m$ is odd.}
\end{align*}
 But if $\omega+\omega$ does not exist, this well-ordering is not
 isomorphic to any ordinal. So $\Z$ does \emph{not} prove
 \olref[ordinals][ordtype]{thmOrdinalRepresentation}. 

Flipping things around: Replacement allows us to prove the existence
of $\omega+\omega$, and hence must allow us to prove the existence of
$V_{\omega+\omega}$. And not just that. For \emph{any} well-ordering
we can define, \olref[ordinals][ordtype]{thmOrdinalRepresentation}
tells us that there is some $\alpha$ isomorphic with that
well-ordering, and hence that $V_\alpha$ exists. In a straightforward
way, then, Replacement guarantees that the hierarchy of sets must be
\emph{very tall}. 

Over the next few sections, and then again in
\olref[card-arithmetic][fix]{sec}, we'll get a better sense of better
just \emph{how} tall Replacement forces the hierarchy to be. The
simple point, for now, is that Replacement really \emph{does} stand in
need of justification!

\end{document}