\documentclass[../../../include/open-logic-section]{subfiles}

\begin{document}

\olfileid{sth}{z}{milestone}
\olsection{$\Z^-$: a Milestone}

We will revisit \stagesinf{} in the next section. However, with the
Axiom of Infinity, we have reached an important milestone. We now have
all the axioms required for the theory $\Zminus$. In detail:

\begin{defn}
The theory $\Zminus$ has these axioms: Extensionality, Union, Pairs, Powersets, Infinity, and all instances of the Separation scheme.
\end{defn}

The name stands for \emph{Zermelo} set theory (\emph{minus} something
which we will come to later). Zermelo deserves the honour, since he
essentially formulated this theory in his
\citeyear{Zermelo1908Untersuchungen}.\footnote{For interesting
comments on the history and technicalities, see \citet[Appendix
A]{Potter2004}.}

This theory is powerful enough to allow us to do an enormous amount of
mathematics. In particular, you \emph{should} look back through
\olref[sfr][][]{part}, and convince yourself that everything we did,
na\"ively, could be done more formally within~$\Zminus$. (Once you
have done that for a bit, you might want to skip ahead and read
\olref[sth][z][arbintersections]{sec}.) So, henceforth,
and without any further comment, we will take ourselves to be working
in $\Zminus$ (at least).

\end{document}