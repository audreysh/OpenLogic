% Part: propositional-logic
% Chapter: syntax-and-semantics
% Section: introduction

\documentclass[../../../include/open-logic-section]{subfiles}

\begin{document}

\olfileid{pl}{syn}{int}

\olsection{Introduction}

Propositional logic deals with !!{formula}s that are built from
!!{propositional variable}s using the propositional connectives
$\lnot$, $\land$, $\lor$, $\lif$, and $\liff$.  Intuitively,
!!a{propositional variable}~$p$ stands for a sentence or proposition
that is true or false. Whenever the ``truth value'' of the
!!{propositional variable} in !!a{formula} is determined, so is the
truth value of any !!{formula}s formed from them using propositional
connectives.  We say that propositional logic is \emph{truth
  functional}, because its semantics is given by functions of truth
values. In particular, in propositional logic we leave out of
consideration any further determination of truth and falsity, e.g.,
whether something is necessarily true rather than just contingently
true, or whether something is known to be true, or whether something
is true now rather than was true or will be true.  We only consider
two truth values true ($\True$) and false ($\False$), and so exclude
from discussion the possibility that a statement may be neither true
nor false, or only half true. We also concentrate only on connectives where
the truth value of !!a{formula} built from them is completely
determined by the truth values of its parts (and not, say, on its
meaning). In particular, whether the truth value of conditionals in
English is truth functional in this sense is contentious. The material
conditional~$\lif$ is; other logics deal with conditionals that are
not truth functional.

In order to develop the theory and metatheory of truth-functional
propositional logic, we must first define the syntax and semantics of
its expressions.  We will describe one way of constructing
!!{formula}s from !!{propositional variable}s using the connectives.
Alternative definitions are possible. Other systems will choose
different symbols, will select different sets of connectives as
primitive, and will use parentheses differently (or even not at all,
as in the case of so-called Polish notation).  What all approaches
have in common, though, is that the formation rules define the set of
!!{formula}s \emph{inductively}. If done properly, every expression
can result essentially in only one way according to the formation
rules.  The inductive definition resulting in expressions that are
\emph{uniquely readable} means we can give meanings to these
expressions using the same method---inductive definition.

Giving the meaning of expressions is the domain of semantics.  The
central concept in semantics for propositonal logic is that of
satisfaction in !!a{valuation}. !!^a{valuation}~$\pAssign{v}$ assigns
truth values $\True$, $\False$ to the !!{propositional variable}s. Any
!!{valuation} determines a truth value $\pValue{v}(!A)$ for any
!!{formula}~$!A$.  !!^a{formula} is satisfied in
!!a{valuation}~$\pAssign{v}$ iff $\pValue{v}(!A) = \True$---we write
this as $\pSat{v}{!A}$. This relation can also be defined by induction on
the structure of~$!A$, using the truth functions for the logical
connectives to define, say, satisfaction of $!A \land !B$ in terms of
satisfaction (or not) of $!A$ and~$!B$.

On the basis of the satisfaction relation $\pSat{v}{!A}$ for sentences
we can then define the basic semantic notions of tautology,
entailment, and satisfiability.  !!^a{formula} is a tautology,
$\Entails !A$, if every !!{valuation} satisfies it, i.e.,
$\pValue{v}(!A) = \True$ for any~$\pAssign{v}$. It is entailed by a
set of !!{formula}s, $\Gamma \Entails !A$, if every !!{valuation} that
satisfies all the !!{formula}s in~$\Gamma$ also satisfies~$!A$. And a
set of !!{formula}s is satisfiable if some !!{valuation} satisfies all
!!{formula}s in it at the same time.  Because !!{formula}s are
inductively defined, and satisfaction is in turn defined by induction
on the structure of !!{formula}s, we can use induction to prove
properties of our semantics and to relate the semantic notions
defined.


\end{document}
