% Part: computability
% Chapter: machines-computations
% Section: halting-problem

\documentclass[../../../include/open-logic-section]{subfiles}

\begin{document}

\olfileid{tms}{und}{hal}
\olsection{The Halting Problem}

\begin{explain}
Assume we have fixed some finite descriptions of Turing machines.
Using these, we can enumerate Turing machines via their descriptions,
say, ordered by the lexicographic ordering.  Each Turing machine thus
receives an \emph{index}: its place in the enumeration $M_1$, $M_2$,
$M_3$, \dots of Turing machine descriptions.

We know that there must be non-Turing-computable functions: the set of
Turing machine descriptions---and hence the set of Turing machines---is
enumerable, but the set of all functions from $\Nat$ to $\Nat$ is
not. But we can find specific examples of non-computable function as
well. One such function is the halting function.
\end{explain}

\begin{defn}[Halting function]
 The \emph{halting function}~$h$ is defined as
\[
h(e,n) =
\begin{cases}
  \text{0} & \text{if machine~$M_e$ does not halt for input $n$} \\
  \text{1} & \text{if machine~$M_e$ halts for input $n$}
\end{cases}
\]
\end{defn}

\begin{defn}[Halting problem]
The \emph{Halting Problem} is the problem of determining (for any $e$, $n$)
whether the Turing machine~$M_e$ halts for an input of~$n$ strokes.
\end{defn}

\begin{explain}
We show that~$h$ is not Turing-computable by showing that a related
function,~$s$, is not Turing-computable. This proof relies on the fact
that anything that can be computed by a Turing machine can be computed
using just two symbols: $\TMblank$ and $\TMstroke$, and the fact that two
Turing machines can be hooked together to create a single machine.
\end{explain}

\begin{defn} The function~$s$ is defined as
\[
s(e) =
\begin{cases}
  \text{0} & \text{if machine~$M_e$ does not halt for input $e$} \\
  \text{1} & \text{if machine~$M_e$ halts for input $e$}
\end{cases}
\]
\end{defn}

\begin{lem}
The function~$s$ is not Turing computable.
\end{lem}

\begin{proof}
We suppose, for contradiction, that the function~$s$ is
Turing computable.  Then there would be a Turing machine~$S$ that
computes~$s$. We may assume, without loss of generality, that when $S$
halts, it does so while scanning the first square.  This machine can
be ``hooked up'' to another machine~$J$, which halts if it is started on a
blank tape (i.e., if it reads $\TMblank$ in the initial state while
scanning the square to the right of the end-of-tape symbol), and
otherwise wanders off to the right, never halting. $S \concat J$, the
machine created by hooking $S$ to $J$, is a Turing machine, so it is
$M_e$ for some~$e$ (i.e., it appears somewhere in the enumeration). Start $M_e$
on an input of~$e$ $\TMstroke$s. There are two possibilities: either
$M_e$ halts or it does not halt.
\begin{enumerate}
\item Suppose $M_e$ halts for an input of $e$ $\TMstroke$s. Then $s(e)
  = 1$. So $S$, when started on~$e$, halts with a single $\TMstroke$
  as output on the tape.  Then $J$ starts with a $\TMstroke$ on the
  tape. In that case $J$ does not halt. But $M_e$ is the machine $S
  \concat J$, so it should do exactly what $S$ followed by $J$ would
  do.  So $M_e$ cannot halt for an input of $e$ $\TMstroke$'s.

\item Now suppose $M_e$ does not halt for an input of $e$
  $\TMstroke$s.  Then $s(e) = 0$, and $S$, when started on input~$e$,
  halts with a blank tape.  $J$, when started on a blank tape,
  immediately halts.  Again, $M_e$ does what $S$ followed by~$J$ would
  do, so $M_e$ must halt for an input of $e$ $\TMstroke$'s.
\end{enumerate}
This shows there cannot be a Turing machine~$S$: $s$ is not Turing
computable.
\end{proof}

\begin{thm}[Unsolvability of the Halting Problem]
\ollabel{thm:halting-problem} The halting problem is unsolvable, i.e.,
the function~$h$ is not Turing computable.
\end{thm}

\begin{proof}
Suppose $h$ were Turing computable, say, by a Turing machine~$H$.
We could use $H$ to build a Turing machine that computes~$s$: First,
make a copy of the input (separated by a blank). Then move back to the
beginning, and run $H$.  We can clearly make a machine that does the
former, and if $H$ existed, we would be able to ``hook it up'' to such
a modified doubling machine to get a new machine which would determine
if $M_e$ halts on input~$e$, i.e., computes~$s$. But we've already
shown that no such machine can exist. Hence, $h$ is also not Turing
computable.
\end{proof}

\begin{prob}
The Three Halting (3-Halt) problem is the problem of giving a decision
procedure to determine whether or not an arbitrarily chosen Turing
Machine halts for an input of three strokes on an otherwise blank
tape. Prove that the 3-Halt problem is unsolvable.
\end{prob}

\begin{prob}
Show that if the halting problem is solvable for Turing machine and
input pairs $M_e$ and $n$ where $e \neq n$, then it is also solvable
for the cases where $e = n$.
\end{prob}

\begin{prob}
We proved that the halting problem is unsolvable if the input is a
number~$e$, which identifies a Turing machine~$M_e$ via an enumaration
of all Turing machines.  What if we allow the description of Turing
machines from \olref[tur][und][enu]{sec} directly as input? (This
would require a larger alphabet of course.)  Can there be a Turing
machine which decides the halting problem but takes as input
descriptions of Turing machines rather than indices?  Explain why or
why not.
\end{prob}

\end{document}
