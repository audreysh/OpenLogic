% Part: incompleteness
% Chapter: incompleteness-provability
% Section: lob-thm

\documentclass[../../../include/open-logic-section]{subfiles}

\begin{document}

\olfileid{inc}{inp}{tar}

\olsection{The Undefinability of Truth}

The notion of \emph{definability} depends on having a formal semantics
for the language of arithmetic.  We have described a set of formulas
and sentences in the language of arithmetic. The ``intended
interpretation'' is to read such sentences as making assertions about
the natural numbers, and such an assertion can be true or false. Let
$\Struct{N}$ be the !!{structure} with domain $\Nat$ and the standard
interpretation for the symbols in the language of arithmetic.  Then
$\Sat{N}{!A}$ means ``$!A$ is true in the standard interpretation.''

\begin{defn}
A relation $R(x_1,\dots,x_k)$ of natural numbers is \emph{definable}
in $\Struct{N}$ if and only if there is a formula $!A(x_1,\dots,x_k)$
in the language of arithmetic such that for every $n_1,\dots,n_k$,
$R(n_1,\dots,n_k)$ if and only if $\Sat{N}{!A(\num n_1,\dots,\num
  n_k)}$.
\end{defn}

Put differently, a relation is definable in $\Struct{N}$ if and
only if it is representable in the theory $\Th{TA}$, where $\Th{TA} =
\Setabs{!A}{\Sat{N}{!A}}$ is the set of true sentences of
arithmetic. (If this is not immediately clear to you, you should go
back and check the definitions and convince yourself that this is the
case.)

\begin{lem}
Every computable relation is definable in~$\Struct{N}$.
\end{lem}

\begin{proof}
It is easy to check that the formula representing a relation in
$\Th{Q}$ defines the same relation in $\Struct{N}$. 
\end{proof}

Now one can ask, is the converse also true?  That is, is every
relation definable in~$\Struct{N}$ computable? The answer is no. For
example:

\begin{lem}
The halting relation is definable in $\Struct{N}$.
\end{lem}

\begin{proof}
Let $H$ be the halting relation, i.e.,
\[
H = \Setabs{\tuple{e,x}}{\lexists[s][T(e, x, s)]}.
\]
Let $!D_T$ define $T$ in $\Struct{N}$. Then
\[
H = \Setabs{\tuple{e,x}}{\Sat{N}{\lexists[s][!D_T(\num e, \num x, s)]}},
\]
so $\lexists[s][!D_T(z, x, s)]$ defines~$H$ in $\Struct{N}$. 
\end{proof}

\begin{prob}
Show that $Q(n) \defiff n \in \Setabs{\Gn{!A}}{\Th{Q} \Proves !A}$ is
  definable in arithmetic.
\end{prob}

What about $\Th{TA}$ itself? Is it definable in arithmetic? That
is: is the set $\Setabs{\Gn{!A}}{\Sat{N}{!A}}$ definable in
arithmetic? Tarski's theorem answers this in the negative.

\begin{thm}
\ollabel{thm:tarski}
The set of true !!{sentence}s of arithmetic is not definable in arithmetic.
\end{thm}

\begin{proof} 
Suppose $!D(x)$ defined it, i.e., $\Sat{N}{!A}$ iff
$\Sat{N}{!D(\gn{!A})}$. By the fixed-point lemma, there is a formula
$!A$ such that $\Th{Q} \Proves !A \liff \lnot !D(\gn{!A})$, and hence
$\Sat{N}{!A \liff \lnot !D(\gn{!A})}$. But then $\Sat{N}{!A}$ if and
only if $\Sat{N}{\lnot !D(\gn{!A})}$, which contradicts the fact that
$!D(y)$ is supposed to define the set of true statements of
arithmetic.  
\end{proof}

Tarski applied this analysis to a more general philosophical notion of
truth. Given any language $L$, Tarski argued that an adequate notion
of truth for $L$ would have to satisfy, for each sentence $X$,
\begin{quote}
`$X$' is true if and only if $X$.
\end{quote}
Tarski's oft-quoted example, for English, is the sentence
\begin{quote}
`Snow is white' is true if and only if snow is white.
\end{quote}
However, for any language strong enough to represent the diagonal
function, and any linguistic predicate $T(x)$, we can construct a
sentence $X$ satisfying ``$X$ if and only if not $T(\text{`$X$'})$.''
Given that we do not want a truth predicate to declare some sentences
to be both true and false, Tarski concluded that one cannot specify a
truth predicate for all sentences in a language without, somehow,
stepping outside the bounds of the language. In other words, a the
truth predicate for a language cannot be defined in the language
itself.

\end{document}
