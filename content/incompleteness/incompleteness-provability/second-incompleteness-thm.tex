% Part: incompleteness
% Chapter: incompleteness-provability
% Section: second-incompleteness-thm

\documentclass[../../../include/open-logic-section]{subfiles}

\begin{document}

\olfileid{inc}{inp}{2in}

\olsection{The Second Incompleteness Theorem}

How can we express the assertion that $\Th{PA}$ doesn't prove its own
consistency? Saying $\Th{PA}$ is inconsistent amounts to saying that
$\Th{PA} \Proves \eq[0][1]$. So we can take the consistency statement
$\OCon[\Th{PA}]$ to be the !!{sentence} $\lnot
\OProv[\Th{PA}](\gn{\eq[0][1]})$, and then the following theorem does
the job:

\begin{thm}
\ollabel{thm:second-incompleteness} 
Assuming $\Th{PA}$ is consistent, then $\Th{PA}$ does not !!{derive}
$\OCon[\Th{PA}]$.
\end{thm}

It is important to note that the theorem depends on the particular
representation of $\OCon[\Th{PA}]$ (i.e., the particular
representation of $\OProv[\Th{PA}](y)$). All we will use is that the
representation of $\OProv[\Th{PA}](y)$ satisfies the three
!!{derivability} conditions, so the theorem generalizes to any theory
with !!a{derivability} predicate having these properties.

It is informative to read G\"odel's sketch of an argument, since the
theorem follows like a good punch line. It goes like this. Let
$!G_\Th{PA}$ be the G\"odel sentence that we constructed in the proof
of \olref[1in]{thm:first-incompleteness}. We have shown ``If $\Th{PA}$
is consistent, then $\Th{PA}$ does not !!{derive} $!G_\Th{PA}$.'' If we
formalize this \emph{in} $\Th{PA}$, we have a proof of
\[
\OCon[\Th{PA}] \lif \lnot \OProv[\Th{PA}](\gn{!G_\Th{PA}}).
\]
Now suppose $\Th{PA}$ !!{derive}s $\OCon[\Th{PA}]$. Then it !!{derive}s $\lnot
\Prov[\Th{PA}](\gn{!G_\Th{PA}})$. But since $!G_\Th{PA}$ is a G\"odel
sentence, this is equivalent to $!G_\Th{PA}$. So $\Th{PA}$ !!{derive}s
$!G_\Th{PA}$.

But: we know that if $\Th{PA}$ is consistent, it doesn't !!{derive}
$!G_\Th{PA}$!{}  So if $\Th{PA}$ is consistent, it can't !!{derive}
$\OCon[\Th{PA}]$.

To make the argument more precise, we will let $!G_\Th{PA}$ be the
G\"odel sentence for~$\Th{PA}$ and use the !!{derivability} conditions
(P1)--(P3) to show that $\Th{PA}$ !!{derive}s $\OCon[\Th{PA}] \lif
!G_\Th{PA}$. This will show that $\Th{PA}$ doesn't !!{derive}
$\OCon[\Th{PA}]$. Here is a sketch of the proof, in~$\Th{PA}$. (For
simplicity, we drop the $\Th{PA}$ subscripts.)
\begin{align}
& !G \liff \lnot \OProv(\gn{!G}) \ollabel{G2-1}\\
& \qquad\text{$!G$ is a G\"odel sentence}\notag \\
& !G \lif \lnot \OProv(\gn{!G}) \ollabel{G2-2}\\
  & \qquad\text{from \olref{G2-1}} \notag\\
& !G \lif
  (\OProv(\gn{!G}) \lif \lfalse) \ollabel{G2-3}\\
  & \qquad\text{from \olref{G2-2} by logic}\notag\\
& \OProv(\gn{
    !G \lif
    (\OProv(\gn{!G}) \lif \lfalse)
  }) \ollabel{G2-4}\\
  & \qquad\text{by from \olref{G2-3} by condition P1} \notag\\
& \OProv(\gn{!G}) \lif
  \OProv(\gn{
    (\OProv(\gn{!G}) \lif \lfalse)
    }) \ollabel{G2-5}\\
  & \qquad\text{from \olref{G2-4} by condition P2} \notag\\
& \OProv(\gn{!G}) \lif (\OProv(\gn{\OProv(\gn{!G})}) \lif \OProv(\gn{\lfalse})) \ollabel{G2-6}\\
  & \qquad\text{from \olref{G2-5} by condition P2 and logic} \notag\\
& \OProv(\gn{!G}) \lif 
  \OProv(\gn{\OProv(\gn{!G})}) \ollabel{G2-7}\\
   & \qquad\text{by P3} \notag\\
& \OProv(\gn{!G}) \lif \OProv(\gn{\lfalse}) \ollabel{G2-8}\\
  & \qquad \text{from \olref{G2-6} and \olref{G2-7} by logic}\notag\\
& \OCon \lif \lnot \OProv(\gn{!G}) \ollabel{G2-9}\\
  & \qquad\text{contraposition of \olref{G2-8} and $\OCon \ident \lnot \OProv(\gn{\lfalse})$}\notag \\
& \OCon \lif !G \notag\\
  & \qquad\text{from \olref{G2-1} and \olref{G2-9} by logic}\notag
\end{align}
The use of logic in the above just elementary facts from propositional
logic, e.g., \olref{G2-3} uses $\Proves \lnot!A \liff (!A\lif
\lfalse)$ and \olref{G2-8} uses $!A \lif (!B \lif !C), !A \lif !B
\Proves !A \lif !C$. The use of condition~P2 in \olref{G2-5} and
\olref{G2-6} relies on instances of~P2, $\OProv(\gn{!A \lif !B}) \lif
(\OProv(\gn{!A}) \lif \OProv(\gn{!B}))$. In the first one, $!A \ident
!G$ and $!B \ident \OProv(\gn{!G}) \lif \lfalse$; in the second, $!A
\ident \OProv(\gn{G})$ and $!B \ident \lfalse$.

The more abstract version of the second incompleteness theorem is as follows:

\begin{thm}
\ollabel{thm:second-incompleteness-gen} Let $\Th{T}$ be any
consistent, !!{axiomatized} theory extending $\Th{Q}$ and let
$\OProv[T](y)$ be any formula satisfying !!{derivability} conditions
P1--P3 for~$\Th{T}$. Then $\Th{T}$ does not !!{derive}~$\OCon[T]$.
\end{thm}

\begin{prob}
Show that $\Th{PA}$ !!{derive}s $!G_{\Th{PA}} \lif \OCon[\Th{PA}]$.
\end{prob}

\begin{digress}
The moral of the story is that no ``reasonable'' consistent theory for
mathematics can !!{derive} its own consistency statement. Suppose
$\Th{T}$ is a theory of mathematics that includes $\Th{Q}$ and
Hilbert's ``finitary'' reasoning (whatever that may be). Then, the
whole of $\Th{T}$ cannot !!{derive} the consistency statement of
$\Th{T}$, and so, a fortiori, the finitary fragment can't !!{derive}
the consistency statement of~$\Th{T}$ either. In that sense, there
cannot be a finitary consistency proof for ``all of mathematics.''

There is some leeway in interpreting the term ``finitary,'' and
G\"odel, in the 1931 paper, grants the possibility that something we
may consider ``finitary'' may lie outside the kinds of mathematics
Hilbert wanted to formalize. But G\"odel was being charitable; today,
it is hard to see how we might find something that can reasonably be
called finitary but is not formalizable in, say, $\Th{ZFC}$,
Zermelo-Fraenkel set theory with the axiom of choice.
\end{digress}

\end{document}
