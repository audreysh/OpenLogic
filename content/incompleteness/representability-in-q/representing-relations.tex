% Part: incompleteness
% Chapter: representability-in-q
% Section: representing-relations

\documentclass[../../../include/open-logic-section]{subfiles}

\begin{document}

\olfileid{inc}{req}{rel}

\olsection{Representing Relations}

Let us say what it means for a \emph{relation} to be representable.

\begin{defn}
\ollabel{defn:representing-relations} A relation $R(x_0,\dots,x_k)$ on
the natural numbers is {\em representable in $\Th{Q}$} if there is a
formula $!A_R(x_0,\dots,x_k)$ such that whenever $R(n_0,\dots,n_k)$ is
true, $\Th{Q}$ proves $!A_R(\num{n_0},\dots,\num{n_k})$, and whenever
$R(n_0,\dots,n_k)$ is false, $\Th{Q}$ proves $\lnot !A_R(\num{n_0},
\dots, \num{n_k})$.
\end{defn}

\begin{thm}
\ollabel{thm:representing-rels} A relation is representable in
$\Th{Q}$ if and only if it is computable.
\end{thm}

\begin{proof}
For the forwards direction, suppose $R(x_0,\dots,x_k)$ is
represented by the formula $!A_R(x_0,\dots,x_k)$. Here is an
algorithm for computing $R$: on input $n_0$, \dots,~$n_k$, simultaneously
search for a proof of $!A_R(\num{n_0}, \dots, \num{n_k})$ and a proof of
$\lnot !A_R(\num{n_0}, \dots, \num{n_k})$. By our hypothesis, the search
is bound to find one or the other; if it is the first, report ``yes,''
and otherwise, report ``no.''

In the other direction, suppose $R(x_0, \dots, x_k)$ is computable. By
definition, this means that the function $\Char{R}(x_0, \dots, x_k)$
is computable. By \olref[int]{thm:representable-iff-comp}, $\Char{R}$
is represented by a formula, say $!A_{\Char{R}}(x_0, \dots, x_k,
y)$. Let $!A_R(x_0, \dots, x_k)$ be the formula $!A_{\Char{R}}(x_0,
\dots, x_k, \num{1})$. Then for any $n_0$, \dots,~$n_k$, if $R(n_0,
\dots, n_k)$ is true, then $\Char{R}(n_0, \dots, n_k) = 1$, in which
case $\Th{Q}$ proves $!A_{\Char{R}}(\num{n_0}, \dots, \num{n_k},
\num{1})$, and so $\Th{Q}$ proves $!A_R(\num{n_0}, \dots,
\num{n_k})$. On the other hand, if $R(n_0, \dots, n_k)$ is false, then
$\Char{R}(n_0, \dots, n_k) = 0$. This means that $\Th{Q}$ proves
\[
\lforall[y][(!A_{\Char{R}}(\num{n_0}, \dots, \num{n_k}, y) \lif y =
  \num{0})].
\]
Since $\Th{Q}$ proves $\eq/[\num{0}][\num{1}]$, $\Th{Q}$ proves
$\lnot !A_{\Char{R}}(\num{n_0}, \dots, \num{n_k}, \num{1})$, and so it
proves $\lnot !A_R(\num{n_0}, \dots, \num{n_k})$.
\end{proof}

\begin{prob}
Show that if $R$ is representable in~$\Th{Q}$, so is~$\Char{R}$.
\end{prob}

\end{document}
