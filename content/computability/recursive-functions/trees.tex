% Part: computability
% Chapter: recursive-functions
% Section: trees

\documentclass[../../../include/open-logic-section]{subfiles}

\begin{document}

\olfileid{cmp}{rec}{tre}
\olsection{Trees}

Sometimes it is useful to represent trees as natural numbers, just
like we can represent sequences by numbers and properties of and
operations on them by primitive recursive relations and functions on
their codes.  We'll use sequences and their codes to do this. A tree
can be either a single node (possibly with a label) or else a node
(possibly with a label) connected to a number of subtrees. The node is
called the \emph{root} of the tree, and the subtrees it is connected
to its \emph{immediate subtrees}.

We code trees recursively as a sequence $\tuple{k, d_1, \dots, d_k}$,
where $k$ is the number of immediate subtrees and $d_1$, \dots,~$d_k$
the codes of the immediate subtrees. If the nodes have labels, they
can be included after the immediate subtrees. So a tree consisting
just of a single node with label~$l$ would be coded by $\tuple{0,l}$,
and a tree consisting of a root (labelled~$l_1$) connected to two
single nodes (labelled $l_2$, $l_3$) would be coded by $\tuple{2,
  \tuple{0, l_2}, \tuple{0, l_3}, l_1}$.

\begin{prop}
  \ollabel{prop:subtreeseq}
  The function $\fn{SubtreeSeq}(t)$, which returns the code of a
  sequence the elements of which are the codes of all subtrees of the
  tree with code~$t$, is primitive recursive.
\end{prop}

\begin{proof}
  First note that $\fn{ISubtrees}(t) = \fn{subseq}(t, 1, (t)_0)$ is
  primitive recursive and returns the codes of the immediate subtrees
  of a tree~$t$. Now we can define a helper function
  $\fn{hSubtreeSeq}(t,n)$ which computes the sequence of all subtrees
  which are $n$ nodes removed from the root. The sequence of subtrees
  of~$t$ which is $0$ nodes removed from the root---in other words,
  begins at the root of~$t$---is the sequence consisting just
  of~$t$. To obtain a sequence of all level~$n+1$ subtrees of $t$, we
  concatenate the level $n$ subtrees with a sequence consisting of all
  immediate subtrees of the level $n$ subtrees. To get a list of all
  these, note that if $f(x)$ is a primitive recursive function
  returning codes of sequences, then $g_f(s, k) = f((s)_0) \concat
  \dots \concat f((s)_k)$ is also primitive recursive:
    \begin{align*}
      g(s, 0) & = f((s)_0)\\
      g(s, k+1) & = g(s, k) \concat f((s)_{k+1})
    \end{align*}
    For instance, if $s$ is a sequence of trees, then
    $h(s) = g_{\fn{ISubtrees}}(s, \len{s})$ gives the sequence of the
    immediate subtrees of the elements of~$s$. We can use it to define
    $\fn{hSubtreeSeq}$ by
    \begin{align*}
      \fn{hSubtreeSeq}(t, 0) & = \tuple{t} \\
      \fn{hSubtreeSeq}(t, n+1) & = \fn{hSubtreeSeq}(t, n) \concat
      h(\fn{hSubtree}(t, n)).
    \end{align*}
    The maximum level of subtrees in a tree coded by~$t$, i.e., the
    maximum distance between the root and a leaf node, is bounded by
    the code~$t$. So a sequence of codes of all subtrees of the tree
    coded by~$t$ is given by $\fn{hSubtreeSeq}(t, t)$.
\end{proof}

\begin{prob}
  The definition of $\fn{hSubtreeSeq}$ in the proof of
  \olref[cmp][rec][tre]{prop:subtreeseq} in general includes
  repetitions. Give an alternative definition which guarantees that
  the code of a subtree occurs only once in the resulting list.
\end{prob}

\end{document}

