% Part: computability
% Chapter: computability-theory
% Section: k-1

\documentclass[../../../include/open-logic-section]{subfiles}

\begin{document}

\olfileid{cmp}{thy}{k1}
\olsection{An Example of Reducibility}

Let us consider an application of \olref[ppr]{prop:reduce}.

\begin{prop}
\ollabel{prop:k1}
Let
\[
K_1 = \Setabs{e}{\cfind{e}(0) \fdefined}.
\]
Then $K_1$ is computably enumerable but not computable.
\end{prop}

\begin{proof}
Since $K_1 = \Setabs{e}{\lexists[s][T(e,0,s)]}$, $K_1$ is computably
enumerable by \olref[eqc]{thm:exists-char}.

To show that $K_1$ is not computable, let us show that $K_0$ is
reducible to it.

\begin{explain}
This is a little bit tricky, since using $K_1$ we can
only ask questions about computations that start with a particular
input, $0$. Suppose you have a smart friend who can answer questions
of this type (friends like this are known as ``oracles''). Then
suppose someone comes up to you and asks you whether or not $\tuple{e,
  x}$ is in $K_0$, that is, whether or not machine $e$ halts on input
$x$. One thing you can do is build another machine, $e_x$, that, for
\emph{any} input, ignores that input and instead runs~$e$ on input
$x$. Then clearly the question as to whether machine $e$ halts on
input $x$ is equivalent to the question as to whether machine $e_x$
halts on input $0$ (or any other input). So, then you ask your friend
whether this new machine, $e_x$, halts on input $0$; your friend's
answer to the modified question provides the answer to the original
one. This provides the desired reduction of $K_0$ to~$K_1$.
\end{explain}

Using the universal partial computable function, let $f$
be the 3-ary function defined by
\[
f(x,y,z) \simeq \cfind{x}(y).
\]
Note that $f$ ignores its third input entirely. Pick an index $e$ such
that $f = \cfind{e}[3]$; so we have
\[
\cfind{e}[3](x,y,z) \simeq \cfind{x}(y).
\]
By the $s$-$m$-$n$ theorem, there is a function $s(e,x,y)$ such that, for
every $z$,
\begin{align*}
\cfind{s(e,x,y)}(z) & \simeq \cfind{e}[3](x,y,z) \\
& \simeq \cfind{x}(y).
\end{align*}

\begin{explain}
In terms of the informal argument above, $s(e,x,y)$ is an index for
the machine that, for any input $z$, ignores that input and computes
$\cfind{x}(y)$.
\end{explain}

In particular, we have
\[
\cfind{s(e,x,y)}(0) \fdefined \quad \text{if and only if} \quad
\cfind{x}(y) \fdefined.
\]
In other words, $\tuple{x, y} \in K_0$ if and only if $s(e,x,y) \in
K_1$. So the function $g$ defined by
\[
g(w) = s(e,(w)_0,(w)_1)
\]
is a reduction of $K_0$ to~$K_1$.
\end{proof}

\end{document}
