% Part: sets-functions-relations
% Chapter: sets
% Section: russells-paradox

\documentclass[../../../include/open-logic-section]{subfiles}


\begin{document}

\olfileid{sfr}{set}{rus}
\olsection{Russell's Paradox}

Extensionality licenses the notation $\Setabs{x}{\phi(x)}$, for
\emph{the} set of $x$'s such that~$\phi(x)$. However, all that
extensionality \emph{really} licenses is the following thought.
\emph{If} there is a set whose members are all and only the $\phi$'s,
\emph{then} there is only one such set. Otherwise put: having fixed
some~$\phi$, the set $\Setabs{x}{\phi(x)}$ is unique, \emph{if it
exists}.

But this conditional is important!{} Crucially, not every property
lends itself to \emph{comprehension}. That is, some  properties do
\emph{not} define sets. If they all did, then we would run into
outright contradictions. The most famous example of this is Russell's
Paradox.

Sets may be !!{element}s of other sets---for instance, the power set
of a set~$A$ is made up of sets. And so it makes sense to ask or
investigate whether a set is !!a{element} of another set. Can a set be
a member of itself?  Nothing about the idea of a set seems to rule
this out. For instance, if \emph{all} sets form a collection of
objects, one might think that they can be collected into a single
set---the set of all sets. And it, being a set, would be !!a{element}
of the set of all sets. 

Russell's Paradox arises when we consider the property of not having
itself as !!a{element}, of being \emph{non-self-membered}. What if we
suppose that there is a set of all sets that do not have themselves as
!!a{element}? Does
\[
R = \Setabs{x}{x \notin x}
\]
exist? It turns out that we can prove that it does not.

\begin{thm}[Russell's Paradox]\ollabel{thm:russells-paradox}
	There is no set $R = \Setabs{x}{x \notin x}$.
\end{thm}

\begin{proof}
For reductio, suppose that $R = \Setabs{x}{x \notin x}$ exists. Then
$R \in R$ iff $R \notin R$, since sets are extensional.
But this is a contradicion.
\end{proof}

\begin{tagblock}{novice}
\begin{explain}
Let's run through the proof that no set $R$ of non-self-membered sets
can exist more slowly. If $R$ exists, it makes sense to ask if $R \in
R$ or not---it must be either $\in R$ or $\notin R$. Suppose the
former is true, i.e., $R \in R$. $R$~was defined as the set of all
sets that are not !!{element}s of themselves, and so if $R \in R$,
then $R$ does not have this defining property of~$R$. But only sets
that have this property are in~$R$, hence, $R$ cannot be !!a{element}
of~$R$, i.e., $R \notin R$. But $R$ can't both be and not be
!!a{element} of~$R$, so we have a contradiction.

Since the assumption that $R \in R$ leads to a contradiction, we have
$R \notin R$. But this also leads to a contradiction!{} For if $R
\notin R$, it does have the defining property of~$R$, and so would be
!!a{element} of $R$ just like all the other non-self-membered sets.
And again, it can't both not be and be !!a{element} of~$R$.
\end{explain}
\end{tagblock}

\begin{digress}
How do we set up a set theory which avoids falling into
Russell's Paradox, i.e., which avoids making the \emph{inconsistent}
claim that $R = \Setabs{x}{x \notin x}$ exists? Well, we would need to
lay down axioms which give us very precise conditions for stating when
sets exist (and when they don't). 
	
The set theory sketched in this chapter doesn't do this. It's
\emph{genuinely na\"ive}. It tells you only that sets obey
extensionality and that, if you have some sets, you can form their
union, intersection, etc. It is possible to develop set theory more
rigorously than this. \oliflabeldef{cumul:::part}{That rigour will be
reserved for Part \olref[cumul][][]{part}. For now, we will proceed
na\"ively, and carefully try to sidestep contradictions.}{}
\end{digress}


\end{document}
