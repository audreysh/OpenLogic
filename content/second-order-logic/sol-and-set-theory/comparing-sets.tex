% Part: second-order-logic
% Chapter: sol-and-sets
% Section: comparing-sets

\documentclass[../../../include/open-logic-section]{subfiles}

\begin{document}

\olfileid{sol}{set}{cmp}

\olsection{Comparing Sets}

\begin{prop}
The !!{formula} $\lforall[x][(X(x) \lif Y(x))]$ defines the subset
relation, i.e., $\Sat{M}{\lforall[x][(X(x) \lif Y(x))]}[s]$ iff $s(X)
\subseteq s(Y)$.
\end{prop}

\begin{prop}
The !!{formula} $\lforall[x][(X(x) \liff Y(x))]$ defines the identity
relation on sets, i.e., $\Sat{M}{\lforall[x][(X(x) \liff Y(x))]}[s]$
iff $s(X) = s(Y)$.
\end{prop}

\begin{prop}
The !!{formula} $\lexists[x][X(x)]$ defines the property of being
non-empty, i.e., $\Sat{M}{\lexists[x][X(x)]}[s]$ iff $s(X) \neq
\emptyset$.
\end{prop}

A set~$X$ is no larger than a set~$Y$, $\cardle{X}{Y}$, iff there is
!!a{injective} function $f\colon X \to Y$.  Since we can express that
a function is injective, and also that its values for arguments in~$X$
are in~$Y$, we can also define the relation of being no larger than on
subsets of the domain.

\begin{prop}
The formula
\[
\lexists[u][(\lforall[x][(X(x) \lif Y(u(x)))] \land \lforall[x][\lforall[y][(\eq[u(x)][u(y)] \lif
      \eq[x][y])]])]
\]
defines the relation of being no larger than.
\end{prop}

Two sets are the same size, or ``equinumerous,'' $\cardeq{X}{Y}$, iff
there is !!a{bijective} function~$f\colon X \to Y$.

\begin{prop}
The formula
\begin{multline*}
  \lexists[u][(\lforall[x][(X(x) \lif Y(u(x)))] \land {}\\
    \lforall[x][\lforall[y][(\eq[u(x)][u(y)] \lif \eq[x][y])]]
  \land {} \\\lforall[y][(Y(y) \lif \lexists[x][(X(x)
      \land \eq[y][u(x)])])])]
\end{multline*}
defines the relation of being equinumerous with.
\end{prop}

We will abbreviate these !!{formula}s, respectively, as $X \subseteq Y$,
$X = Y$, $X \neq \emptyset$, $\cardle{X}{Y}$, and
$\cardeq{X}{Y}$. (This may be slightly confusing, since we use the
same notation when we speak informally about sets $X$ and $Y$---but
here the notation is an abbreviation for !!{formula}s in second-order
logic involving one-place relation variables $X$ and~$Y$.)

\begin{prop}
The !!{sentence} $\lforall[X][\lforall[Y][((\cardle{X}{Y} \land
    \cardle{Y}{X}) \lif \cardeq{X}{Y})]]$ is valid.
\end{prop}

\begin{proof}
The !!{sentence} is satisfied in !!a{structure}~$\Struct{M}$ if, for
any subsets $X \subseteq \Domain{M}$ and $Y \subseteq \Domain{M}$, if
$\cardle{X}{Y}$ and $\cardle{Y}{X}$ then $\cardeq{X}{Y}$.  But this
holds for \emph{any} sets $X$ and $Y$---it is the Schr\"oder-Bernstein
Theorem.
\end{proof}


\end{document}
