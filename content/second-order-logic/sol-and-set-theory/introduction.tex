% Part: second-order-logic
% Chapter: sol-and-sets
% Section: introduction

\documentclass[../../../include/open-logic-section]{subfiles}

\begin{document}

\olfileid{sol}{set}{int}

\olsection{Introduction}

Since second-order logic can quantify over subsets of the domain as
well as functions, it is to be expected that some amount, at least, of
set theory can be carried out in second-order logic. By ``carry out,''
we mean that it is possible to express set theoretic properties and
statements in second-order logic, and is possible without any special,
non-logical vocabulary for sets (e.g., the membership !!{predicate} of
set theory).  For instance, we can define unions and intersections of
sets and the subset relationship, but also compare the sizes of sets,
and state results such as Cantor's Theorem.

\end{document}
