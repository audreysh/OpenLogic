% Part: first-order-logic
% Chapter: natural-deduction
% Section: quantifier-rules

\documentclass[../../../include/open-logic-section]{subfiles}

\begin{document}

\olfileid{fol}{ntd}{qrl}

\olsection{Quantifier Rules}

\subsection{Rules for $\lforall$}

\begin{defish}
\AxiomC{$!A(a)$}
\RightLabel{\Intro{\lforall}}
\UnaryInfC{$\lforall[x][\Atom{!A}{x}]$}
\DisplayProof
\hfill
\AxiomC{$\lforall[x][\Atom{!A}{x}]$}
\RightLabel{\Elim{\lforall}}
\UnaryInfC{$!A(t)$}
\DisplayProof
\end{defish}

In the rules for~$\lforall$, $t$ is a ground term (a term that does
not contain any variables), and $a$~is a !!{constant} which does not
occur in the conclusion~$\lforall[x][!A(x)]$, or in any assumption
which is !!{undischarged} in the !!{derivation} ending with the
premise~$!A(a)$. We call $a$ the \emph{eigenvariable} of the
\Intro{\lforall} inference.

\subsection{Rules for $\lexists$}

\begin{defish}
\AxiomC{$\Atom{!A}{t}$}
\RightLabel{\Intro{\lexists}}
\UnaryInfC{$\lexists[x][\Atom{!A}{x}]$}
\DisplayProof
\hfill
\AxiomC{$\lexists[x][\Atom{!A}{x}]$}
\AxiomC{[$\Atom{!A}{a}$]$^n$}
\DeduceC{$!C$}
\DischargeRule{\Elim{\lexists}}{n}
\BinaryInfC{$!C$}
\DisplayProof
\end{defish}

Again, $t$ is a ground term, and $a$ is a constant which does not
occur in the premise $\lexists[x][!A(x)]$, in the conclusion~$!C$, or
any assumption which is !!{undischarged} in the !!{derivation}s ending
with the two premises (other than the assumptions $!A(a)$).  We call
$a$ the \emph{eigenvariable} of the \Elim{\lexists} inference.

The condition that an eigenvariable neither occur in the premises nor in
any assumption that is !!{undischarged} in the !!{derivation}s leading 
to the premises for the \Intro{\lforall} or \Elim{\lexists} inference is 
called the \emph{eigenvariable condition}.

\begin{explain}
We use the term ``eigenvariable'' even though $a$ in the above rules
is a constant. This has historical reasons.

In \Intro{\lexists} and \Elim{\lforall} there are no restrictions, and
the term~$t$ can be anything, so we do not have to worry about any
conditions. On the other hand, in the \Elim{\lexists} and
\Intro{\lforall} rules, the eigenvariable condition requires that the
!!{constant}~$a$ does not occur anywhere in the conclusion or in an
!!{undischarged} assumption. The condition is necessary to ensure that
the system is sound, i.e., only !!{derive}s !!{sentence}s from
!!{undischarged} assumptions from which they follow. Without this
condition, the following would be allowed:
\begin{prooftree}
  \AxiomC{$\lexists[x][!A(x)]$}
  \AxiomC{$\Discharge{!A(a)}{1}$}
  \RightLabel{*\Intro{\lforall}}
  \UnaryInfC{$\lforall[x][!A(x)]$}
  \RightLabel{\Elim{\lexists}}
  \BinaryInfC{$\lforall[x][!A(x)]$}
\end{prooftree}
However, $\lexists[x][!A(x)] \Entails/ \lforall[x][!A(x)]$.
\end{explain}


\end{document}
