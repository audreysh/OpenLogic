% Part: first-order-logic
% Chapter: natural-deduction
% Section: propositional-rules

\documentclass[../../../include/open-logic-section]{subfiles}

\begin{document}

\iftag{FOL}
      {\olfileid{fol}{ntd}{prl}}
      {\olfileid{pl}{ntd}{prl}}

\olsection{Propositional Rules}


\subsection{Rules for $\land$}

\begin{defish}
\AxiomC{$!A$}
\AxiomC{$!B$}
\RightLabel{\Intro{\land}}
\BinaryInfC{$!A \land !B$}
\DisplayProof
\hfill
\begin{tabular}{r}
\AxiomC{$!A \land !B$}
\RightLabel{\Elim{\land}}
\UnaryInfC{$!A$}
\DisplayProof
\\[3ex]
\AxiomC{$!A \land !B$}
\RightLabel{\Elim{\land}}
\UnaryInfC{$!B$}
\DisplayProof
\end{tabular}
\end{defish}

\subsection{Rules for $\lor$}

\begin{defish}
\begin{tabular}{r}
\AxiomC{$!A$}
\RightLabel{\Intro{\lor}}
\UnaryInfC{$!A \lor !B$}
\DisplayProof
\\[3ex]
\AxiomC{$!B$}
\RightLabel{\Intro{\lor}}
\UnaryInfC{$!A \lor !B$}
\DisplayProof
\end{tabular}
\hfill
\AxiomC{$!A \lor !B$}
\AxiomC{$\Discharge{!A}{n}$}
\DeduceC{$!C$}
\AxiomC{$\Discharge{!B}{n}$}
\DeduceC{$!C$}
\DischargeRule{\Elim{\lor}}{n}
\TrinaryInfC{$!C$}
\DisplayProof
\end{defish}

\subsection{Rules for $\lif$}

\begin{defish}
\AxiomC{$\Discharge{!A}{n}$}
\DeduceC{$!B$}
\DischargeRule{\Intro{\lif}}{n}
\UnaryInfC{$!A \lif !B$}
\DisplayProof
\hfill
\AxiomC{$!A \lif !B$}
\AxiomC{$!A$}
\RightLabel{\Elim{\lif}}
\BinaryInfC{$!B$}
\DisplayProof
\end{defish}

\subsection{Rules for $\lnot$}

\begin{defish}
\AxiomC{$\Discharge{!A}{n}$}
\noLine
\DeduceC{$\lfalse$}
\DischargeRule{\Intro{\lnot}}{n}
\UnaryInfC{$\lnot !A$}
\DisplayProof
\hfill
\AxiomC{$\lnot !A$}
\AxiomC{$!A$}
\RightLabel{\Elim{\lnot}}
\BinaryInfC{$\lfalse$}
\DisplayProof
\end{defish}

\subsection{Rules for $\lfalse$}

\begin{defish}
\AxiomC{$\lfalse$}
\RightLabel{\FalseInt}
\UnaryInfC{$!A$}
\DisplayProof
\hfill
\AxiomC{$\Discharge{\lnot !A}{n}$}
\DeduceC{$\lfalse$}
\DischargeRule{\FalseCl}{n}
\UnaryInfC{$!A$}
\DisplayProof
\end{defish}

Note that $\Intro{\lnot}$ and $\FalseCl$ are very similar: The
difference is that $\Intro{\lnot}$ derives a negated
!!{sentence}~$\lnot !A$ but $\FalseCl$ a positive !!{sentence}~$!A$.

Whenever a rule indicates that some assumption may be discharged, we
take this to be a permission, but not a requirement. E.g., in the
$\Intro{\lif}$ rule, we may discharge any number of assumptions of the
form~$!A$ in the !!{derivation} of the premise~$!B$, including zero.

\end{document}
