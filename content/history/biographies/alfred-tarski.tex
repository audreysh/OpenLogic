% Part: history
% Chapter: biographies 
% Section: alfred-tarski

\documentclass[../../../include/open-logic-section]{subfiles}

\begin{document}

\olfileid{his}{bio}{tar}

\olsection{Alfred Tarski}

\olphoto{tarski-alfred}{Alfred Tarski}

Alfred Tarski was born on January 14, 1901 in Warsaw, Poland (then
part of the Russian Empire). Described as ``Napoleonic,'' Tarski was
boisterous, talkative, and intense.  His energy was often reflected in
his lectures---he once set fire to a wastebasket while disposing of a
cigarette during a lecture, and was forbidden from lecturing in that
building again.

Tarski had a thirst for knowledge from a young age. Although later in
life he would tell students that he studied logic because it was the
only class in which he got a~B, his high school records show that he
got A's across the board---even in logic. He studied at the University
of Warsaw from 1918 to 1924. Tarski first intended to study
biology, but became interested in mathematics, philosophy, and logic,
as the university was the center of the Warsaw School of Logic and
Philosophy. Tarski earned his doctorate in 1924 under the supervision of
Stanis\l{}aw Le\'{s}niewski.

Before emigrating to the United States in 1939, Tarski completed some
of his most important work while working as a secondary school teacher
in Warsaw. His work on logical consequence and logical truth were
written during this time. In 1939, Tarski was visiting the United
States for a lecture tour. During his visit, Germany invaded Poland,
and because of his Jewish heritage, Tarski could not return.  His wife
and children remained in Poland until the end of the war, but were
then able to emigrate to the United States as well.  Tarski taught at
Harvard, the College of the City of New York, and the Institute for
Advanced Study at Princeton, and finally the University of California,
Berkeley. There he founded the multidisciplinary program in Logic and
the Methodology of Science.  Tarski died on October 26, 1983 at the
age of 82.

\begin{reading} 
For more on Tarski's life, see the biography \emph{Alfred Tarski: Life
  and Logic} \citep{Feferman2004}. Tarski's seminal works on logical
consequence and truth are available in English in \citep{Tarski1983}.
All of Tarski's original works have been collected into a four volume
series, \citep{Tarski1981}.
\end{reading}

\end{document}
